\documentclass[9pt]{jsarticle}
%\usepackage{otf} %Windows上ではコメントアウト
\usepackage{amsmath, amssymb}
\usepackage[dvipdfmx]{graphics,xcolor}
\usepackage{tikz}
\usepackage[framemethod=tikz]{mdframed}
\usepackage{type1cm}
\usepackage{graphicx}
\usepackage{float}
\usepackage{here}
\usepackage{fancyhdr}
\usepackage{lscape} %ページ全体を用いた横向き画像使用時に使用

%マージン設定
%横
\setlength{\textwidth}{165truemm}
\setlength{\hoffset}{-1truein}
\setlength{\oddsidemargin}{25truemm}
%縦
\setlength{\textheight}{257truemm}
\setlength{\voffset}{-1truein}
\setlength{\topmargin}{10truemm}
\setlength{\headheight}{5truemm}
\setlength{\headsep}{5truemm}

%ページ番号設定
\pagestyle{fancy}
\fancyhead[RE]{}
\fancyhead[RO]{\thepage}
\fancyhead[LE]{\thepage}
\fancyhead[LO]{}
\cfoot{}
\renewcommand{\headrulewidth}{0pt}

%図表番号設定
%「図4.4.1」みたいにしたかったらsectionをsubsectionに変える

%箇条書き表示設定
\renewcommand{\labelenumi}{(\arabic{enumi})}%第1階層
\renewcommand{\labelenumii}{(\roman{enumii})}%第2階層

%タイトル設定
\title{4EC 情報通信プロジェクト データ解析\\ 人工知能によるサッカーくじ予想}
\date{2017/05/09}
\author{14EC020 内田奏 \and 14EC052 小林将輝 \and 14EC086 中村文香}

\begin{document}
\maketitle
\thispagestyle{fancy}
\section{概要}
サッカーくじ toto は,独立行政法人日本スポーツ振興センターにより運営・発売が行われている公営ギャンブルである.
totoは国内外で行われるサッカーの試合を対象とし,購入者は勝ち,負け,引き分けを予想する.
ここでは,totoの主要な対象であるJリーグの試合結果をもとに,ニューラルネットワークを用いてこの目を予想する.

\section{収集・使用するデータ}
Jリーグの試合結果はすべてJリーグ公式のデータサイト(https://data.j-league.or.jp/SFTP01/)より取得可能である.
取得したデータから勝ち点,順位等を計算し,それらを入力データとする.
データ収集・整形の分担は次を予定している.
\begin{table}[htbp]
\centering
\caption{データ収集・整形 分担表} \label{Equipments}
\begin{tabular}{cc} 
\hline
役割&担当者\\\hline\hline
データ収集&内田\\
データ整形&小林・中村\\\hline
\end{tabular}
\end{table}

\section{使用するツール}
\begin{itemize}
\item{ディープラーニングフレームワーク}
\begin{itemize}
\item{Chainer}
\item{TensorFlow}
\item{Caffe2}
\end{itemize}
\item{プログラミング言語}
\begin{itemize}
\item{Python}
\end{itemize}
\end{itemize}

\section{工程表}
\begin{table}[htbp]
\centering
\caption{工程表(予定)} \label{Equipments}
\begin{tabular}{cc} 
\hline
作業項目&時期\\\hline\hline
データ収集&5月中\\
データ整形&6月中\\
シミュレータ作成&7月中\\
ネットワーク作成&8月以降\\\hline
\end{tabular}
\end{table}

\end{document}
