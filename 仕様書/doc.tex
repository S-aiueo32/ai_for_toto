\documentclass[twocolumn,10pt]{jarticle} 
\usepackage[dvipdfmx]{graphicx, color}
\usepackage[top=15truemm,bottom=20truemm,left=20truemm,right=20truemm]{geometry}
\setlength{\columnsep}{3zw} 
\usepackage[labelsep=period]{caption} 
%\pagestyle{empty}


\makeatletter
\def\section{\@startsection {section}{1}{\z@}%
{2.5ex \@plus -1ex \@minus -.2ex}%
{1ex \@plus.1ex}%
{\sectionformat}}
\def\sectionformat{\normalsize\bf}


\def\@maketitle{% 
 \begin{flushright}% 
 {\@date}% 日付 
\end{flushright}% 
 \begin{center}% 
 {\LARGE \@title \par}% タイトル 
\end{center}% 
 \begin{flushright}% 
 {\large \@author}% 著者 
\end{flushright}% 
 \par\vskip 1.5em
 }
\makeatother


\title{\Large 情報通信プロジェクト データ解析\\ -人工知能によるサッカーくじ予想-} 
\vspace{-1zh}
\author{\normalsize14EC020 内田奏\\14EC052 小林将輝\\14EC086 中村文香} 
\vspace{-1zh}
\date{2017年5月9日(火)}
\begin{document}
\maketitle
\section{概要}
サッカーくじ toto は,独立行政法人日本スポーツ振興センターにより運営・発売が行われている公営ギャンブルである.
totoは国内外で行われるサッカーの試合を対象とし,購入者は勝ち,負け,引き分けを予想する.
主要な対象は国内リーグのJリーグであり,対象試合にはJ1・J2・J3の最大28試合のうち13試合が選ばれる.
ここでは,過去のJリーグの試合結果をもとに,勝ち,負け,引き分けを予測する.

\section{収集・使用するデータ}
Jリーグの試合結果はすべてJリーグ公式のデータサイト(https://data.j-league.or.jp)より取得可能である.
取得したデータを整形し,それらを入力データとする.
データ収集・整形の分担は表\ref{data}を予定している.
\begin{table}[htbp]
\centering
\caption{データ収集・整形 分担表} \label{data}
\begin{tabular}{cc} 
\hline
\textbf{役割}&\textbf{担当者}\\\hline
データ収集&内田\\
データ整形&小林・中村\\\hline
\end{tabular}
\end{table}

Jリーグは,totoが開始された2000年に初めて引き分けを導入し,1999年以前は延長戦およびPK戦により勝敗を決していた.
また,2000年から2002年に導入された引き分けは,延長戦でVゴールが記録されなかった場合にのみ適用されるものであった.
現在では,いかなる場合においても延長戦は行わない.
このように試合方式に相違があるが,ここでは次のように勝ち,負け,引き分けを定義する.
\begin{itemize}
\item{1999年以前の試合結果について}
\begin{itemize}
\item{90分及び延長での得点を勝敗として採用する.}
\item{PK戦に突入した試合は,PK戦の結果にかかわらず引き分けとする.}
\end{itemize}
\item{2000年から2002年の試合結果について}
\begin{itemize}
\item{公式記録の勝敗を採用する.}
\item{Vゴールによる勝利も勝利とする.}
\end{itemize}
\item{2003年以降の試合結果について}
\begin{itemize}
\item{公式記録の勝敗を採用する.}
\end{itemize}
\end{itemize}
これによる勝ち点等公式記録との差異は今回考慮しない.

リーグ方式については,全年度1ステージ制であると仮定し,2ステージ制によるリーグの動向やプレーオフ等については考慮しない.

\section{使用するツール}
学習にはニューラルネットワークを用いる.
ニューラルネットワークを用いたディープラーニングフレームワークおよびツールは多くリリースされているが,
今回は次のフレームワークを用いることを考えている.
\begin{itemize}
\item{TensorFlow}

時系列データの学習に長けたリカレントニューラルネットワーク(RNN)及びLSTMについてのドキュメントや実装例が比較的豊富.

\item{Chainer}

Pythonによる実装が可能.

\item{NVIDIA DIGITS}

CaffeをラップしてGUIで使用可能.画像分類のための知識が筆者に多少ある.

\end{itemize}

\section{工程}
作業工程については,表\ref{process}のスケジュールを予定している.
\begin{table}[htbp]
\centering
\caption{工程表(予定)} \label{process}
\begin{tabular}{cc} 
\hline
\textbf{作業項目}&\textbf{時期}\\\hline
データ収集&5月中\\
データ整形&6月中\\
シミュレータ作成&7月中\\
ネットワーク作成&8月中\\
シミュレーション実行&上記が終わり次第\\\hline
\end{tabular}
\end{table}

\end{document}
